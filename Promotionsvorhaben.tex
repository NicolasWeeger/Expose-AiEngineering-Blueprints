\documentclass[11pt,a4paper,pointlessnumbers]{scrartcl}

\usepackage[utf8]{inputenc}
\usepackage[T1]{fontenc}
\usepackage[ngerman]{babel}
\usepackage{lmodern}
\usepackage{fancyhdr}
\usepackage[pdftex]{graphicx}
\usepackage{paralist}
\usepackage{listings}
\usepackage{color}
\usepackage{mathtools}
\usepackage{amsmath,amssymb}
\usepackage{framed,pstricks}
\usepackage[framed]{ntheorem}
\usepackage{geometry}
\usepackage{tabularx}
\usepackage{setspace}
\usepackage{array}
\usepackage[german,algoruled,vlined,longend]{algorithm2e}
\usepackage[labelfont=bf, font={footnotesize,stretch=1.2,sf}]{caption}
\usepackage{float}
\restylefloat{figure}
\usepackage{enumitem}
\usepackage[breaklinks, pdftex]{hyperref}
\usepackage{url}

%\renewcommand\familydefault{\sfdefault}\\
\renewcommand{\familydefault}{\rmdefault}
\setkomafont{disposition}{\rmfamily}
\urlstyle{same}
\allowdisplaybreaks

\geometry{a4paper, left=2cm, right=2cm, top=2.5cm, bottom=3cm} 

% Farben definieren
\definecolor{dkgreen}{rgb}{0,0.6,0}
\definecolor{gray}{rgb}{0.5,0.5,0.5}
\definecolor{mauve}{rgb}{0.58,0,0.82}
\definecolor{codegray}{rgb}{0.5,0.5,0.5}
\definecolor{backcolour}{rgb}{0.95,0.95,0.92}
\definecolor{dkred}{rgb}{0.54,0,0}
%\definecolor{shadecolor}{gray}{.95}

% Abstand Blattkante zu Kopf- und Fußzeile
\setlength{\footskip}{1.5cm} 

% Erste Zeile eines Absatzes nicht einrücken
\setlength{\parindent}{0pt}

% Literaturverzeichnis Zitate auf Nummern einstellen
\bibliographystyle{ieeetr}

% Zeilenabstand einstellen
\setstretch{1.15}

% Schriftart arial
\renewcommand{\rmdefault}{phv} % Arial
\renewcommand{\sfdefault}{phv} % Arial

% Schriftgröße 11pt
\renewcommand{\normalsize}{\fontsize{11pt}{13.2pt}\selectfont}

% Listings einstellen
\lstset{
	language = matlab,
	numbers=left,
	numberstyle=\tiny\color{codegray},
	numbersep=5pt,
	breaklines=true,
	showstringspaces=false,
	frame=l ,
	xleftmargin=10pt,
	xrightmargin=0pt,
	basicstyle=\ttfamily\scriptsize,
	stepnumber=1,
	deletekeywords = {beta,gamma},
	keywordstyle=\color{blue},          % keyword style
  	commentstyle=\color{dkgreen},       % comment style
  	stringstyle=\color{dkred},          % string literal style
  	tabsize=3,
  	emphstyle=\color{blue},
  	emph={fft2, ifft2}
}

% Mathematische Umgebungen mit fortlaufender, vom Kapitel abhängiger Nummerierung
% Beweise
\theoremheaderfont{\normalfont\itshape}
\theorembodyfont{\normalfont}
\newtheorem*{Beweis}{Beweis.}
\newtheorem*{Beweisskizze}{Beweisskizze.}

\theorempreskip{17pt}
\theorempostskip{17pt}

\theoremheaderfont{\normalfont\bfseries}
\theorembodyfont{\normalfont}
\theoremstyle{break}
\newtheorem{Satz}{Satz}[section]
\newtheorem{Definition}[Satz]{Definition}   
\newtheorem{Lemma}[Satz]{Lemma}	
\newtheorem{Proposition}[Satz]{Proposition}
\newtheorem{Beispiel}[Satz]{Beispiel}
\newtheorem{Bemerkung}[Satz]{Bemerkung}
\newtheorem{Algorithmus}[Satz]{Algorithmus}
\newtheorem{Modell}[Satz]{Modell}
\newtheorem{Korollar}[Satz]{Korollar}
                  
\numberwithin{equation}{section} 

% Wichtige Abkürzungen
\newcommand{\C}{\mathbb{C}} % komplexe
\newcommand{\K}{\mathbb{K}} % Körper
\newcommand{\R}{\mathbb{R}} % reelle
\newcommand{\Q}{\mathbb{Q}} % rationale
\newcommand{\Z}{\mathbb{Z}} % ganze
\newcommand{\N}{\mathbb{N}} % natuerliche
\newcommand{\sgn}{\operatorname{sgn}} % Signum-Funktion
\newcommand*{\qed}{\hfill\ensuremath{\square}} % QED
\newcommand{\prox}{\operatorname{Prox}}
\newcommand{\relint}{\operatorname{relint}}
\newcommand{\Spann}{\operatorname{span}}
\newcommand{\aff}{\operatorname{aff}}
\newcommand{\F}{\mathcal{F}}
\newcommand{\A}{\mathcal{A}}
\newcommand{\D}{\mathcal{D}}
\newcommand{\J}{\mathcal{J}}
\renewcommand{\H}{\mathcal{H}}
\renewcommand{\L}{\mathcal{L}}
\newcommand{\X}{\mathcal{X}}
\newcommand{\Y}{\mathcal{Y}}
\renewcommand{\div}{\operatorname{div}}
\newcommand{\dom}{\operatorname{dom}}
\newcommand{\Bild}{\operatorname{Bild}}
\newcommand{\Kern}{\operatorname{Kern}}
\newcommand{\rank}{\operatorname{Rang}}
\newcommand{\Eig}{\operatorname{Eig}}
\renewcommand{\Re}{\mathcal{R}\mathnormal{e}}
\renewcommand{\Im}{\mathcal{I}\mathnormal{m}}
\DeclareMathOperator*{\argmin}{arg\,min}
\DeclareMathOperator*{\argmax}{arg\,max}
\renewcommand{\labelenumi}{\textit{\roman{enumi})}}

% Tabelle mit fester Breite linksbündig, zentriert, rechtsbündig
\newcolumntype{L}[1]{>{\raggedright\arraybackslash}p{#1}}
\newcolumntype{C}[1]{>{\centering\arraybackslash}p{#1}}
\newcolumntype{R}[1]{>{\raggedleft\arraybackslash}p{#1}}

\newcommand\undermat[2]{%
  \makebox[0pt][l]{$\smash{\underbrace{\phantom{%
    \begin{matrix}#2\end{matrix}}}_{\text{$#1$}}}$}#2}

\begin{document}
\pagestyle{fancy}
\fancyhf{}
\fancyhead[L]{\textit{\nouppercase{\leftmark}}}
\fancyhead[R]{\thepage}
\date{}
\title{\textbf{Dissertationsvorhaben von Nicolas Weeger} 
\\{Thema: AI Engineering Blueprints for practicable Machine Learning development}
\\{\normalsize Hochschule für angewandte Wissenschaften Ansbach, Fakultät Technik}
\\{\normalsize Erstbetreuer: Prof. Dr. Christian Uhl}\vspace{-.5em}
\\{\normalsize Zweitbetreuer: Prof. Dr. Stefan Geißelsöder}
\\{\normalsize angestrebter Titel: Dr. rer. nat.}}
\maketitle
\vspace{-14ex}

\thispagestyle{empty}
\section{Forschungsthema} \label{kap:motivation}

Künstliche Intelligenz (KI) verändert zahlreiche Industrien
und Anwendungsbereiche. Für Unternehmen ist es entscheidend
KI-Techniken einzusetzen, um geschäftlichen Erfolg zu erzielen
[1], [2]. Vor allem KMU können von der Einführung von
Einsatz von KI-Techniken profitieren. Sie können ihre Fähigkeiten
Fähigkeiten in bestimmten Bereichen, wie z.B. Kundenerfahrung,
Produktionsüberwachung und Entscheidungsprozesse [3].
Die firmeneigene Entwicklung von ML-Modellen zur Verwendung als
oder innerhalb eines Produkts, genannt KI-Systeme, in einem organisatorischen
Kontext kann zu einer Reihe von Herausforderungen führen [4]-[7]. Diese
Dazu gehört das Verständnis für die Feinheiten der KI, einschließlich
ihre funktionalen Anforderungen und Einsatzszenarien. Die
Integration zusätzlicher Prozesse, wie z. B. Datengenerierung
Datengenerierung und -vorverarbeitung oder Modelltraining und -einsatz, in den
traditionellen Softwareentwicklungsprozess kann potenziell zu organisatorischen
 Promblemen führen, insbesondere für KMU [6]. Der ML
Modellentwicklungszyklus umfasst zusätzliche Praktiken
neben DevOps für die Daten und Modelle. Dazu gehören MLOps
und DataOps-Techniken, die eine Kultur, Praktiken
und Werkzeuge für den Umgang mit Daten und Modellen. Darüber hinaus muss die
Systemarchitektur für KI-Systeme auf die Anforderungen des
Anforderungen des zugrunde liegenden Modells abgestimmt sein. Training und Inferenz
Trainings- und Inferenzumgebungen, sowie Datenspeicherung und Versionierung der
der verschiedenen Artefakte müssen integriert werden, damit das
System als KI-System zu funktionieren.
Folglich erfordert die effektive Implementierung von KI-Systemen
eine gut durchdachte Architektur, die auf die speziellen Anforderungen
Anforderungen der beabsichtigten KI-Anwendung zugeschnitten ist. Diese Dissertation
beschäftigt sich mit der Entwicklung von Blueprints, die auf die Anforderungen
der verschiedenen Arten von KI und Entwicklungsstufen abgestimmt sind. Diese Blueprints verbinden 
die Prinzipien von AI-Engineering, DevOps, MLOps und DataOps um die Herausforderungen, die mit der 
Entwicklung von KI-Systemen verbunden sind, zu bewältigen. Folglich werden sie
die Blueprints anwenden können, indem sie Referenzarchitekturen
Referenzarchitekturen und geeignete Automatisierungsansätze für verschiedene
Arten von KI implementieren.


\section{Stand der Forschung}
\subsection{AI Engineering}

AI-Engineering ist eine Weiterentwicklung des Bereichs Software
Software-Engineering, und aufgrund des raschen Wachstums der ML-Entwicklungen
ist es ein aufstrebendes Feld. Die KI-Technik befindet sich derzeit auf dem
Höhepunkt der Erwartungen“, wie der von Gartner
Gartners AI Hype Cycle für 2024 (https://www.gartner.com/en/articles/hype-cycle-for-artifcial-intelligence)
Engineering ist die Grundlage für die unternehmensweite Bereitstellung von KI und
GenAI in großem Umfang. Den meisten Unternehmen fehlen die Daten-, Analytik
und Software-Grundlagen, um einzelne KI-Projekte in großem
Produktion zu bringen - geschweige denn ein Portfolio von KI
Lösungen in großem Umfang zu betreiben.“ Mehr als ein Dutzend Projekte wurden untersucht in
[8] untersucht, bei denen die Herausforderungen der KI-Technik zu Problemen bei der
ML-Modelle zu produzieren. Die Studie ergab, dass die Mehrheit
Unternehmen, die Modelle für maschinelles Lernen entwickeln, auf
Schwierigkeiten haben, wenn sie versuchen, diese in die Produktion zu überführen. 

Sie bieten eine Forschungsagenda und einen Überblick über die Fragen, die
die in dieser Richtung angegangen werden müssen. [9] weisen darauf hin, dass zwar
der Bereich der Software-Engineering-Forschung bereits ausgiebig
diskutiert wurde, KI-Engineering jedoch viel weniger behandelt wurde. Nur
eine begrenzte Anzahl von Publikationen präsentieren konkrete Erfahrungen
Erfahrungen mit der Anwendung von KI-Engineering-Prinzipien. Sie
wählten zehn AI-Engineering-Praktiken in mehreren Kategorien aus der
aus der Literatur und wendeten sie auf eine Beispielimplementierung an
um die Praktiken und ihre Systemarchitektur zu bewerten.
Darüber hinaus haben Gespräche und Fragebögen, insbesondere
und Fragebögen, insbesondere mit KMUs, haben gezeigt, dass der Wunsch besteht, KI
KI in ihre Systeme zu implementieren. Allerdings hängt der Erfolg ihrer Modell
Modells hängt jedoch von der Bewältigung der oben genannten
oben genannten Herausforderungen ab. So kann die Anwendung von KI-Engineering
kann Unternehmen helfen, die Entwicklung, den Einsatz
und den Betrieb von Modellen für maschinelles Lernen zu optimieren.

\subsection{MLOps}
Die Idee von MLOps ist die Bereitstellung von Techniken und Werkzeugen für
den Einsatz und Betrieb von KI-Systemen bereitzustellen [10]. Das Ziel
ist es, eine Strategie für die Lösung von realen Problemen mit
den Einsatz von ML-Modellen. Mehrere Studien untersuchen verschiedene
Literatur in diesem Bereich und bieten Pipelines, Taxonomien, Werkzeuge,
Methoden und Herausforderungen in diesem Bereich [11]-[13].
[14] führen eine systematische Mapping-Studie für MLOps
Architekturen durch und zeigen 35 Architekturkomponenten auf, de-
beschreiben verschiedene Architekturvarianten für unterschiedliche Anwendungsfälle und
stellen gängige Werkzeuge für diese Architekturkomponenten zur Verfügung.


\subsection{Weitere relevante Forschungsarbeiten}
In [15] wurde eine Referenzarchitektur für die spezifischen Anwendungsfälle in der
der Prozessindustrie vorgestellt, die sich mit Edge Devices befasst.
Sie demonstrierten die Architektur durch die Implementierung einer Fallstudie
Fallstudie für einen realen Anwendungsfall und bewies die Funktionalität
mit dieser Anwendung.
[16] entwickelte eine Referenzarchitektur zur Erleichterung der Nutzung
von Big Data in Edge Computing ML-Techniken zu erleichtern. Sie
verschiedene Ansichten über die Architektur der Modellentwicklung
Entwicklung und Einsatz für diesen speziellen Anwendungsfall.
Eine andere Studie [17] stellt eine Vision für „disziplinierte, wiederholbare
wiederholbare und transparente modellgetriebene Entwicklung und Machine
Learning Operations (MLOps) von intelligenten Unternehmensapplikationen
cations“. Sie stellen ein dreistufiges Metamodell für die modell
modellbasierte Entwicklung von AI/ML-Blueprints auf Basis intelligenter
Anwendungsarchitektur.

Mit Blick auf Software und Architektur werden Entwurfsmuster für
KI-basierte Systeme in mehreren Studien diskutiert [18]-[21].
Sie geben einen Überblick über Entwurfsmuster, die für KI-Anwendungsfälle angepasst oder
für KI-Anwendungsfälle, und zeigen die Anwendung und die
und die daraus resultierenden Vorteile bei der Entwicklung von Modellen für maschinelles
Modelle.

\subsection{Zusammenfassung des Standes der Forschung}
Zusammenfassend lässt sich sagen, dass die Literatur Einblicke in die Wichtigkeit
Bedeutung, mögliche Architekturen und Prinzipien für KI-Engineering
und MLOps-Praktiken. Allerdings ist die Anwendung dieser
Erkenntnisse konzentriert sich derzeit auf einige wenige Referenzarchitekturen
in bestimmten Bereichen, wie z. B. Big Data oder Edge Devices. Andere
Studien konzentrieren sich auf die Definition von Architekturen und Mustern und beweisen
ihre Anwendbarkeit in Fallstudien.
Die Prinzipien des AI-Engineering bilden die Grundlage
die Grundlage für die Entwicklung der in diesem Papier vorgeschlagenen Blaupausen.
MLOps-Pipelines und -Tools sowie bestehende Referenzarchitekturen
Referenzarchitekturen und Frameworks werden eingesetzt, um die Entwicklung
Entwicklung von KI-Systemen zu unterstützen und so den Prozess zu rationalisieren, zu standardisieren
und beschleunigen den Prozess. Software- und Architekturentwurfsmuster
werden zur Beschreibung der Entwicklung verwendet, um die
um die nicht-funktionalen Anforderungen (NFRs) für die verschiedenen
Entwürfe zu erfüllen. Der Einsatz in Feldprojekten ermöglicht ein flexibles,
hochautomatisierten Einsatz und ressourcenschonenden Betrieb
für unterschiedliche Anforderungen in KMUs.



\section{Ziele und wissenschaftlicher Beitrag der Promotion}


\section{Methodik}
Um die Entwürfe zu untersuchen und ihre Brauchbarkeit für Unternehmen zu validieren
für Unternehmen zu überprüfen, wird die Methode der Design Science Research (DSR)
[22], [23], eingesetzt werden. Mit Hilfe von Interviews und einer
einer umfassenden Literaturrecherche werden die Herausforderungen und
Anforderungen von KMU identifiziert und Verbesserungspotenziale
Verbesserungsmöglichkeiten ermittelt.
Basierend auf diesen Erkenntnissen können Geschäftsanforderungen
in Zusammenarbeit mit den relevanten Interessengruppen festgelegt werden,
die sich an deren spezifischen Bedürfnissen orientieren. Durch die Integration der fachlichen
Anforderungen mit den Anforderungen der verschiedenen Arten von
der eingesetzten KI, z. B. Algorithmen, Datenspeicherung, Kom
und NFRs, kann ein umfassendes Rahmenwerk
Rahmenwerk entwickelt werden, das die Verifizierung der entworfenen
Artefakte zu erleichtern. Anschließend können diese einer iterativen
Tests und Validierung unterzogen werden. Schließlich können die Artefakte in den
den Projekten der Beteiligten als Feldtest eingesetzt werden. Dieser Prozess wird dann
Dieser Prozess wird solange wiederholt, bis die Anforderungen fnalisiert sind und die
Artefakte die Anforderungen nachweislich erfüllen. Der Prozess wird
in mehreren Projekten wiederholt, um die Ergebnisse zu verallgemeinern und
sie für KMU so anwendbar wie möglich zu machen.


\section{Arbeits- und Zeitplan}

\section{Zuordnung zum Promotionskolleg REDIG}


% \newpage
\bibliography{deepeeg}

\end{document}

